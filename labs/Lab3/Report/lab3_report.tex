\documentclass{article}
\usepackage{iclr2016_conference,times}

\usepackage{float}
\usepackage{hyperref}
\usepackage{url}
\usepackage{amsmath,amssymb}
\usepackage{natbib}
\usepackage{wrapfig}
\usepackage{graphicx}
\usepackage{subfig}

\bibliographystyle{abbrvnat}


\usepackage{caption}
%\usepackage{subcaption}

\title{
	Mobile Computing (CS23400$/$1) \vspace{-4pt} \\
	{\Large Lab 3 - Report} \vspace{6pt} \\
	{\large Andrea F. Daniele $\hspace{2.2cm}$ Max X. Liu $\hspace{2.2cm}$ Noah A. Hirsch}
}

\begin{document}

\maketitle


\vspace{-1.2cm}

\section{Task and Challenges}
\vspace{-.3cm}
The goal of this lab was to deploy a lane detection algorithm and to design
an algorithm that moves the car along the detected lane.

Although the lane detection algorithm was provided, the hardware and software
provided imposed some key challenges to creating a robust self driving
algorithm. The car itself had difficulty driving in a straight line because the
wheels were skewed slightly towards the right, this means that to drive in a
straight path the car would constantly have to adjust the perceived angle of
its wheels. Furthermore, the camera attached to the car was limited in its range
of vision. This led challenges where the camera was unable to capture the lane
within the frame if the car was driving too quickly or if the turn was too sharp.
On the software wide, the lane detection algorithm was extremely unpredictable
when the lanes were out of frame so a lot of effort was dedicated to making sure
that the camera would always be able to capture both of the lanes. Finally, the
rate at which the car was receiving lane data was quite slow (a few frames a
second), this was a bottleneck in how quickly we could allow the car to drive
itself.

\section{Proposed Approach}
\vspace{-.3cm}
At a high level, the algorithm we used to drive the car consisted of taking the
midpoint of the camera frame and taking the calculated midpoint position of the
detected lanes. Ideally, we would want the midpoint of the detected lanes to match
the midpoint of the camera frame.

In order to accomplish this we first calculated the dimensions of the frame returned
by the camera and set that as the ideal midpoint. Now, for every midpoint returned
by the lane detection algorithm we can calculate the distance between that and the
ideal midpoint. This then gives us an idea of the direction and angle for which we
should point the wheels and should create relatively smooth movement of the car(
no sudden jerky movements to adjust or make a turn).

An additional feature that we had implemented in order to combat the issue
with loss of lane detection is to create a "memory" of previous midpoints and
use a moving average of these to determine what angle to position the wheels. If
the lane readings were to vary drastically this would help to smooth out noisy
readings and keep the car on track.

\section{Results}
\vspace{-.3cm}
We ran tests that varied speed, angle changes, and memory. It was found that the
slower the speed the higher the chance the car would successfully drive itself in
the lane, low to medium angle changes were more affective than aggressive angle changes,
and we found that having no memory (not deploying the memory feature) was more
effective than having a memory.

Due to the slow rate at which the lane detection algorithm was able to send data to
our car, the car had to move very slowly in order to process the lane in front of it
and make adjustments accordingly. If the car moved too quickly, by the time the car
decided to make an adjustment, its actual location was no longer in its perceived location.

When deciding how much to turn each wheel when the midpoint was off, we took an iterative
approach where we tested varying degree change functions (modest, medium, aggressive).
Essentially, this meant changing some scalar value that determined how much to turn
based on the distance from the ideal midpoint.

Finally, it was found that having a memory was not as effective as simply making sure
that the car could keep both lanes in the frame at all times. Having a memory actually
decreased the performance of our car due to the equal weighting of the previous moves,
the car did not turn as much was necessary in order to stay in the lane.

Ultimately, we were able to successfully navigate straight lanes, left turns, and right
turns. This was done by calibrating our midpoint to angle change algorithm until it was
able to complete the course. It also resulted in extremely smooth driving by our car due to
the constant angle adjustments.


\section{Conclusion}
\vspace{-.3cm}
As autonomous vehicles become more and more prevalent, it is not difficult to imagine
a future where the roads are dominated by autonomous vehicles. Through the experiments
that we ran with very simple hardware and software, we were able to relatively easily
achieve self driving within a lane. This was done simply by establishing an ideal point
to follow and making adjustments along the way as the car deviated from the ideal. In modern
autonomous vehicles, the hardware is a lot more powerful and more precise allowing for more
complex maneuvers and situations; however, this was a good first step in understanding
the difficulties of autonomous vehicles and working towards fully understanding autonomous
vehicles.

\bibliographystyle{abbrvnat}
{\scriptsize%
\bibliography{references}
}

\end{document}
