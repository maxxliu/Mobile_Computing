\documentclass{article}
\usepackage{iclr2016_conference,times}

\usepackage{float}
\usepackage{hyperref}
\usepackage{url}
\usepackage{amsmath,amssymb}
\usepackage{natbib}
\usepackage{wrapfig}
\usepackage{graphicx}
\usepackage{subfig}

\bibliographystyle{abbrvnat}


\usepackage{caption}
%\usepackage{subcaption}

\title{
	Mobile Computing (CS23400$/$1) \vspace{-4pt} \\
	{\Large Lab 3 - Report} \vspace{6pt} \\
	{\large Andrea F. Daniele $\hspace{2.2cm}$ Max X. Liu $\hspace{2.2cm}$ Noah A. Hirsch}
}

\begin{document}

\maketitle


\vspace{-1.2cm}

\section{Task and Challenges}
\vspace{-.3cm}
The goal of this lab was to deploy a lane detection algorithm and to design
an algorithm that moves the car along the detected lane.

Although the lane detection algorithm was provided, the hardware and software
provided imposed some key challenges to creating a robust self driving
algorithm. The car itself had difficulty driving in a straight line because the
wheels were skewed slightly towards the right, this means that to drive in a
straight path the car would constantly have to adjust the perceived angle of
its wheels. Furthermore, the camera attached to the car was limited in its range
of vision. This led challenges where the camera was unable to capture the lane
within the frame if the car was driving too quickly or if the turn was too sharp.
On the software wide, the lane detection algorithm was extremely unpredictable
when the lanes were out of frame so a lot of effort was dedicated to making sure
that the camera would always be able to capture both of the lanes. Finally, the
rate at which the car was receiving lane data was quite slow (a few frames a
second), this was a bottleneck in how quickly we could allow the car to drive
itself.

\section{Proposed Approach}
\vspace{-.3cm}
At a high level

\section{Results}
\vspace{-.3cm}
//TODO


\section{Conclusion}
\vspace{-.3cm}
//TODO

\bibliographystyle{abbrvnat}
{\scriptsize%
\bibliography{references}
}

\end{document}
