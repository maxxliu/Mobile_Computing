\documentclass{article}
\usepackage{iclr2016_conference,times}

\usepackage{subfig}
\usepackage{float}
\usepackage{hyperref}
\usepackage{url}
\usepackage{amsmath,amssymb}
\usepackage{natbib}
\bibliographystyle{abbrvnat}

\title{Project Proposal\vspace{-6pt}\\{\large Andrea F. Daniele $\hspace{2cm}$ Max X. Liu $\hspace{2cm}$ Noah A. Hirsch }}

\begin{document}

\maketitle


\vspace{-1.2cm}

\section*{Introduction}
\vspace{-.3cm}
As autonomous vehicles become more prevalent, a large concern is ensuring safety. 
An important consideration for safety is how autonomous vehicles communicate with each other. 
More specifically, we want to investigate the optimal number of vehicles for each vehicle to communicate with. 
Obviously, communicating with as many vehicles as possible would be ideal. 
The problem is that communicating with each vehicle takes time, and the time spent communicating 
with some distant car might be better spent keeping a nearer car more up to date. 
The goal of this project is to study what is the best number of vehicles a self-driving car should 
communicate with in order to maximize safety while minimizing the cost.

\section*{Problem}
\vspace{-.3cm}
In order to increase autonomous vehicle safety and optimize route planning, vehicles must be able to effectively communicate with one another. Specifically, we need to figure out the optimal number, and which, vehicles to communicate with while taking into account the time and cost of communication.

\section*{Challenges}
\vspace{-.3cm}
One of the main challenges in fleet-level communication is the choice of a network infrastructure suitable
for the task. WiFi networks are fast and cheap to deploy but have a limited range. Communication via satellites
offer an unlimited range but it is expensive to deploy. Another challenge is that of deciding how many cars
to share information with and how many cars to get information from. Even though self-driving vehicles are
expected to be capable of processing a lot of information locally, we still need to minimize the impact of a 
fleet-level communication architecture on the on-board computer.

\section*{Proposed Approach and Timeline}
\vspace{-.3cm}




%\bibliographystyle{abbrvnat}
%{\scriptsize%
%\bibliography{references}
%}

\end{document}