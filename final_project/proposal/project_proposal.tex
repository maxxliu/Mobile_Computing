\documentclass{article}
\usepackage{iclr2016_conference,times}

\usepackage{subfig}
\usepackage{float}
\usepackage{hyperref}
\usepackage{url}
\usepackage{amsmath,amssymb}
\usepackage{natbib}
\bibliographystyle{abbrvnat}

\title{Project Proposal\vspace{-6pt}\\{\large Andrea F. Daniele $\hspace{2cm}$ Max X. Liu $\hspace{2cm}$ Noah A. Hirsch }}

\begin{document}

\maketitle


\vspace{-1.2cm}

\section*{Introduction}
\vspace{-.3cm}
As autonomous vehicles become more prevalent, a large concern is ensuring that they are as safe as possible. A big consideration for safety is how autonomous vehicles communicate with each other. 


\section*{Problem}
\vspace{-.3cm}

\section*{Challenges}
\vspace{-.3cm}
One of the main challenges in fleet-level communication is the choice of a network infrastructure suitable
for the task. WiFi networks are fast and cheap to deploy but have a limited range. Communication via satellites
offer an unlimited range but it is expensive to deploy. Another challenge is that of deciding how many cars
to share information with and how many cars to get information from. Even though self-driving vehicles are
expected to be capable of processing a lot of information, we still need to minimize the impact of a fleet-level
communication architecture on the on-board computer.

\section*{Proposed Approach and Timeline}
\vspace{-.3cm}




%\bibliographystyle{abbrvnat}
%{\scriptsize%
%\bibliography{references}
%}

\end{document}