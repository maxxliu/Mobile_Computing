\begin{abstract}
As autonomous vehicles become more prevalent, a large concern is ensuring safety.
The most used perception system on autonomous vehicles is vision. Even though cameras are
cheap and easy to integrate, they are prone to failure in low-visibility scenarios (e.g., foggy weather).
An important consideration for safety is how autonomous vehicles communicate with each other in order
to reduce the risk of collisions in hazardous and unforeseeable situations.
In this work, we want to investigate the optimal communication strategy for self-driving vehicles
in a low-visibility hazardous scenario.
Obviously, communicating with as many vehicles as possible would be ideal, but in practice, the presence of
both physical and technological limitations such as wireless communication range limits and low-bandwidth
communication channels impose a more structured and optimized communication strategy.
Our study identifies some of the critical aspects of fleet-level communication, such as message broadcasting,
message propagation and reaction time aware communication strategies.
Tests on a realistic model of a town show that our communication strategy can reduce the number of
collisions by about $\collision_ratio_reduction_perc\%$.
\end{abstract}
