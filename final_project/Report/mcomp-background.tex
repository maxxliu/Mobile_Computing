\section{Background}

For over a decade, the feasibility and applications of communication between autonomous
vehicles have been well defined and studied. The biggest limitations at the time were
network infrastructure and autonomous vehicle technology, but modern networks and technologies
have made communication applications not only feasible but also necessary. 
The applications of communication between autonomous vehicles
span beyond road vehicles and have many practical benefits to other transportation methods
such as trains and planes, but for the purposes of this work the discussion will center
around road vehicles.

The applications of vehicle communication have been well thought out, but one of the key remaining
challenges is implementing a system infrastructure that is able to
effectively deploy such features while keeping in mind the network requirements~\cite{willke2009survey}.
Features such as emergency warnings, collision avoidance, and motion planning
all require different degrees of network speed and reliability that range from
extreme reliability and low latency to more relaxed requirements. With modern networks,
this is becoming less of an issue but is still a key factor we keep in mind while
simulating our communication network.
Recent work has been done to make
the implementation of communication infrastructure much easier by providing a standardized
framework for coordination among a fleet of vehicles~\cite{keila2018}.
This framework would remove
the complications of needing to program individual vehicles and abstracts the global
state of systems controlling deployed vehicles. This will allow users to quickly build
upon single vehicle tasks and deploy concurrent, sequential, or event based tasks
to a fleet of vehicles. This would be extremely useful in road scenarios that require
a large degree of coordination between vehicles in order to safely and efficiently
navigate (e.g. a large pile up of cars on a highway during rush hour).

Improving the efficiency of vehicles on roads (especially highways) has long been a
very obvious application for autonomous vehicle communication~\cite{murray2007recent}.
There has been both recent and established research in this area, in which the question
was essentially how to best model vehicles following one another to
optimize safety and efficiency~\cite{ou2017extended, tanner2003coordination}.
Primary variables considered are velocity
and distance, with findings that support flocking behavior of vehicles where velocities
and distances between vehicles converge to a common value.
In this work, we focus on studying the safety of different communication models.
Specifically, we look at scenarios where traditional methods may be ineffective and propose
a solution that takes into account variables such as network latency and reaction speed.
We present five unique communication features and detail the effectiveness of each one of them
in the context of collision ratio, and safety distance.